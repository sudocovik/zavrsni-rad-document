\chapter{Zaključak}

Ovaj rad je praktična izrada digitaliziranog sustava kontrole pristupa upotrebnom kreditnih kartica bez narušavanja
privatnosti korisnika - osobni podaci ne dolaze u doticaj sa sustavom.
Upotrebom web platforme, modernih programskih jezika, radnih okvira, biblioteka i oblaka razvijena je aplikacija
za kontrolu pristupa koja omogućava administratorima sustava definiranje popisa kreditnih kartica koje imaju pristup sefu.
Sef se sastoji od nekoliko elektroničkih komponenti i upravljačkog softvera koji koristi web aplikaciju pri
donošenju odluke treba li sef otključati ili ne.

Implementacijom sustava moguće je odgovoriti na pitanja postavljena u uvodnom djelu rada:

\begin{itemize}
    \item Je li kontrolor prisutan?

    S obzirom da su komponente integrirane u sef, da, kontrolor je prisutan.

    \item Jesu li uvjeti stvarno ispunjeni (zloupotreba ovlasti ili ljudska greška)?

    Pod pretpostavkom da je administrator definirao kartice korisnika koji smiju pristupiti sefu, da,
    uspješna autorizacija podrazumijeva pravilno ispunjene uvjete.

    \item Jesu li podaci korisnika ispravni (npr.\ ime i prezime)?

    Podaci o korisnicima sustavu nisu dostupni tako da, ne, ispravnost podataka nije moguće utvrditi.

    \item Jesu li meta podaci ispravni (npr.\ datum i vrijeme)?

    Pod pretpostavkom da nema vanjskog utjecaja (npr.\ greška u javnom NTP serveru, pokvaren interni sat računala), da,
    meta podaci su ispravni.
\end{itemize}

Drugi i četvrti stavak treba uzeti sa zadrškom jer se baziraju na pretpostavkama.
Osim navedenih pretpostavki, također je pretpostavka da programer nije napravio logičke greške tijekom razvoja.
Računalne programe nije moguće dokazati, ali ih je moguće promatrati i testirati.
Pravovaljanim automatiziranim testiranjem i promatranjem (eng. \textit{automated testing and observability}) podiže se
pouzdanost pravilnoga rada sustava.
Navedeni problemi ne bi trebali biti razlog odbacivanja digitaliziranog sustava kontrole pristupa jer
pravilnim dizajniranjem i testiranjem nadmašuje ručni sustav kontrole pristupa.
