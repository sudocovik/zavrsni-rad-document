\chapter{Uvod}

Svakim danom svijet se sve više digitalizira i mehanički procesi se automatiziraju kako bi se uštedjelo na vremenu i resursima.
Sve više smo svjesni koliko računala lako i s manje grešaka obavljaju ponavljajuće zadatke, za razliku od ljudi.
Jedan od ponavljajućih i sklon pogrešci zadatak jest kontrola i evidencija pristupa.
Proces zahtjeva dva sudionika: kontrolora i korisnika.
U tradicionalnom (ne-digitaliziranom) okruženju kontrolor je osoba koja prima zahtjeve korisnika, prema određenim uvjetima
dozvoljava ili odbija pristup korisniku te evidentira u evidencijsku knjigu komu i kad je pristup dozvoljen ili odbijen.
Proces je vrlo jednostavan, no ima više prostora za pogrešku:
\begin{itemize}
    \item Je li kontrolor prisutan?
    \item Jesu li uvjeti stvarno ispunjeni (zloupotreba ovlasti ili ljudska greška)?
    \item Jesu li podaci korisnika ispravni (npr.\ ime i prezime)?
    \item Jesu li meta podaci ispravni (npr.\ datum i vrijeme)?
\end{itemize}
Digitalizacijom ovog procesa kontrolor više nije osoba već uređaj, evidencijska knjiga nije list papira nego oblak (eng. \textit{cloud}).
Ista pitanja mogu biti postavljena nakon digitalizacije, no vjerojatnost pogreške je znatno manja.
