\chapter{Uvod}

Svakim danom svijet se sve više digitalizira i mehanički procesi se automatiziraju kako bi se uštedjelo na vremenu i resursima.
Sve više smo svjesni koliko računala lako i s manje grešaka obavljaju ponavljajuće zadatke, za razliku od ljudi.
Jedan od ponavljajućih i sklon pogrešci zadatak jest kontrola i evidencija pristupa.
Proces zahtjeva dva sudionika: kontrolora i korisnika.
U tradicionalnom (ne-digitaliziranom) okruženju kontrolor je osoba koja prima zahtjeve korisnika, prema određenim uvjetima
dozvoljava ili odbija pristup korisniku te evidentira u evidencijsku knjigu komu i kad je pristup dozvoljen ili odbijen.
Proces je vrlo jednostavan, no ima više prostora za pogrešku:
\begin{itemize}
    \item Je li kontrolor prisutan?
    \item Jesu li uvjeti stvarno ispunjeni (zloupotreba ovlasti ili ljudska greška)?
    \item Jesu li podaci korisnika ispravni (npr.\ ime i prezime)?
    \item Jesu li meta podaci ispravni (npr.\ datum i vrijeme)?
\end{itemize}
Digitalizacijom ovog procesa kontrolor više nije osoba već uređaj, evidencijska knjiga nije list papira nego oblak (eng. \textit{cloud}).
Ista pitanja mogu biti postavljena nakon digitalizacije, no vjerojatnost pogreške je znatno manja.

Praktični dio rada je implementacija kontrole pristupa (eng. \textit{access control}) pomoću potrošačkih elektroničkih
komponenti (eng. \textit{consumer electronics}), oblaka i kreditnih kartica.
Elektroničke komponente tvore uređaj (kontrolor) u obliku sefa s kojim korisnik komunicira dok kreditna kartica služi kao sredstvo identifikacije.
U oblaku se nalazi aplikacija u kojoj se definiraju kreditne kartice koje imaju pravo pristupa.
Uspješni i neuspješni pokušaji pristupanju se evidentiraju u bazu podataka u oblaku.

\section{Hipoteze i ciljevi}

Sljedeće su hipoteze:

\begin{itemize}
    \item \textbf{H1}: Može se pročitati jedinstveni identifikator kreditne kartice bez narušavanja privatnosti (čitanje osobnih podataka).
    \item \textbf{H2}: Moguće je stvoriti kontrolni uređaj pomoću potrošačke elektronike.
    \item \textbf{H3}: Internet protokolima je moguće delegirati logiku autoriziranja na udaljeni server.
\end{itemize}

Ciljevi rada su:

\begin{itemize}
    \item \textbf{C1}: Razviti pločicu s mikrokontrolerom i potrebnim perifernim komponentama.
    \item \textbf{C2}: Smjestiti pločicu i komponente u sef.
    \item \textbf{C3}: Omogućiti konfiguriranje bežične pristupne točke sefa.
    \item \textbf{C4}: Definirati infrastrukturu u oblaku.
    \item \textbf{C5}: Razviti grafičko sučelje na web platformi za upravljanje dozvolama i pregledavanje povijesti otvaranja.
    \item \textbf{C6}: Razviti pozadinsku aplikaciju na web platformi za obrađivanje autorizacijskih zahtjeva sefa.
    \item \textbf{C7}: Ostvariti komunikaciju sefa i udaljenog servera pomoću HTTP protokola.
    \item \textbf{C8}: Zabilježiti svaki pokušaj autoriziranja (uspješan ili neuspješan) na udaljenom serveru.
\end{itemize}
