Pozadinska aplikacija definira većinski dio logike sustava.
Svoje funkcionalnosti daje na korištenje kroz REST API kako bi grafičko sučelje i sef mogli ispunjavati svoje uloge.
Funkcionalnosti su sljedeće:

\begin{itemize}
    \item Upravljanje karticama
    \item Autorizacija kartica
    \item Pregled dnevnika pristupa
    \item Autorizacija korisnika
\end{itemize}

Funkcionalnosti su implementirane u programskom jeziku \textit{PHP} uz pomoć \textit{Laravel} radnog okvira.

\subsection{Upravljanje karticama}

Entitet "Kartica" sadrži podatke o imenu kartice i jedinstvenom identifikatoru (\textit{UID}).
Sadrži i meta podatke o datumu stvaranja i datumu izmjene.
Entitetom korisnik upravlja kroz grafičko sučelje, a pozadinska aplikacija koristi jedinstvene identifikatore pri autoriziranju
zahtjeva sefa.

Entitet kartice nema relacija na druge entitete pa nema smisla prikazivati ER dijagram već je dovoljan programski kod.
Radni okvir omogućava definiranje migracija baze podataka kroz programski jezik umjesto pisanja čistog SQL-a.

\begin{lstlisting}[language=PHP]
class CreateCardsTable extends Migration
{
    public function up()
    {
        Schema::create('cards', function (Blueprint $table) {
            $table->id();
            $table->string('name')->unique();
            $table->string('uid')->unique();
            $table->timestamps();
        });
    }

    public function down()
    {
        Schema::dropIfExists('cards');
    }
}
\end{lstlisting}

Radni okvir ima ORM (\textit{Object Relational Mapper}) zvan \textit{Eloquent} što omogućava rad s bazom kroz objekte
umjesto ručnog pisanja SQL upita.
Dovoljno je definirati klasu prema imenu tablice, u jednini, i naslijediti klasu \textit{Model} iz radnog okvira.

\begin{lstlisting}[language=PHP]
class Card extends Model
{
    static function existsByUid(string $uid)
    {
        return static::query()->where('uid', '=', $uid)->exists() === true;
    }
}
\end{lstlisting}

Statična metoda \textit{existsByUid} provjerava postoji li zapis kartice s danim \textit{UID-em} i vraća \textit{true} ili \textit{false}.
Ona je osnovica za autorizacijske zahtjeve sefa.

\pagebreak

Operacije nad entitetom kartice su izložene (dane na korištenje) vanjskom svijetu, odnosno grafičkom sučelju kroz niz
HTTP putanji:

\begin{table}[h!]
    \centering
    \caption{Prikaz putanja svih funkcionalnosti upravljanja karticama}
    \begin{tabularx}{\textwidth}{|X|X|X|}
        \hline
        \cellcolor{gray!25} Funkcionalnost & \cellcolor{gray!25} HTTP Metoda + Putanja & \cellcolor{gray!25} Klasa@Metoda \\
        \hline
        Popis svih kartica & GET /api/card & CardController@index \\
        \hline
        Stvaranje kartice & POST /api/card & CardController@store \\
        \hline
        Uređivanje kartice & PUT /api/card/\{id\} & CardController@update \\
        \hline
        Brisanje kartice & DELETE /api/card/\{id\} & CardController@destroy \\
        \hline
    \end{tabularx}
    \\[10pt]
    \label{tab:card_functionalities}
\end{table}

Iz tablice~\ref{tab:card_functionalities} može se vidjeti da sve funkcionalnosti dijele isti dio putanje URL-a: \textbf{/api/card}
kao i istu klasu u kojoj su implementirane metode: \textbf{CardController}.
Kako su \textit{Create, Read, Update, Delete (CRUD)} operacije nad istim entitetom ponavljajući uzorak u većini aplikacija, radni okvir omogućava
jednostavno definiranje putanja i pripadnih metoda:

\begin{lstlisting}[language=PHP]
Route::resource('card', CardController::class)->only(['index', 'store', 'update', 'destroy']);
\end{lstlisting}

Prema tablici~\ref{tab:card_functionalities} popis svih kartica je implementiran u metodi \textbf{index}.
Jednostavan upit dohvaća osnovne podatke o karticama, sortira ih po datumu stvaranja i vraća u JSON formatu.

\begin{lstlisting}[language=PHP]
public function index(): JsonResponse
{
    $cardsByCreationDateAscending = Card::query()->select(['id', 'name', 'uid'])->orderBy('created_at')->get();

    return response()->json([
        'data' => $cardsByCreationDateAscending
    ]);
}
\end{lstlisting}

Prema tablici~\ref{tab:card_functionalities} stvaranje nove kartice je implementirano u metodi \textbf{store}.
Metoda stvara objekt \textit{Card} i popunjava ga s podacima dobivenim od strane korisnika.
Kod uspješnog spremanja vraćaju se svi podaci o kartici u JSON formatu, a kod neuspješnog spremanja vraća se kratka poruka,
također u JSON formatu.

\begin{lstlisting}[language=PHP]
public function store(StoreCard $request): JsonResponse
{
    $name = $request->name;
    $uid = $request->uid;

    $card = new Card();
    $card->name = $name;
    $card->uid = $uid;

    if ($card->save()) {
        return response()->json([
            'data' => $card->fresh()
        ]);
    }

    return response()->json(['data' => 'Neuspjesna pohrana kartice. Pokusajte ponovo!']);
}
\end{lstlisting}

Tip argumenta \textit{\$request} je \textit{StoreCard}.
To je klasa koju radni okvir instancira prilikom obrađivanja zahtjeva i ona služi za validaciju podataka.
Ukoliko uvjeti nisu zadovoljeni, metoda \textit{store} se neće izvršiti već se vraća kratka poruka sa svim problemima u
JSON formatu.

\begin{lstlisting}[language=PHP]
class StoreCard extends FormRequest
{
    public function rules(): array
    {
        return [
            'name' => [
                'required',
                'string',
                'max:30',
                Rule::unique(Card::class, 'name')
            ],
            'uid' => [
                'required',
                'string',
                'max:40',
                Rule::unique(Card::class, 'uid')
            ]
        ];
    }

    public function messages(): array
    {
        return [
            'name.unique' => 'Ime vec postoji.',
            'name.max' => 'Ime moze biti maksimalno 30 znakova.',
            'name.*' => 'Ime je obavezno.',
            'uid.unique' => 'UID vec postoji.',
            'uid.max' => 'UID moze biti maksimalno 40 znakova.',
            'uid.*' => 'UID je obavezan.'
        ];
    }
}
\end{lstlisting}

Metoda \textit{rules} vraća polje (eng. \textit{array}) svih uvjeta koje treba zadovoljiti.
Naziv kartice je obavezan (\textit{'required'}) tekst (\textit{'string'}) do trideset znakova (\textit{'max:30'}) i mora biti
unikatan, tj.\ u bazi podataka ne smije postojati kartica s istim imenom (\textit{Rule::unique}).
Slični uvjeti vrijede i za \textit{UID} kartice, razlika je jedino u maksimalnom broju znakova i naziv polja po kojem se
gleda uvjet unikatnosti.
Metoda \textit{messages} vraća polje s porukama korisnim za korisnika kako bi znao u čemu je problem.

\pagebreak

Prema tablici~\ref{tab:card_functionalities} uređivanje postojeće kartice je implementirano u metodi \textbf{update}.
Funkcionalnost je slična kao i kod stvaranja nove kartice, no implementacija je nešto drukčija.

\begin{lstlisting}[language=PHP]
public function update(Card $card, UpdateCard $request): JsonResponse
{
    $card->name = $request->name;

    if ($card->save()) {
        return response()->json([
            'data' => $card->fresh()
        ]);
    }

    return response()->json(['data' => 'Neuspjesno azuriranje kartice. Pokusajte ponovo!']);
}
\end{lstlisting}

Kako se radi o postojećoj kartici potrebno ju je jedinstveno identificirati.
Za to se koristi jedinstveni identifikator odnosno primarni ključ kojeg dodjeljuje baza podataka.
Pri slanju zahtjeva identifikator je dio putanje (\textit{/api/card/\{id\}} - tablica~\ref{tab:card_functionalities}).
Nalazi se u vitičastim zagradama pa radni okvir zna da vrijednost mora proslijediti u pripadajuću metodu (\textit{update}).
Zaglavlje metode bi onda izgledalo:

\begin{lstlisting}[language=PHP]
public function update(int $id, UpdateCard $request): JsonResponse
\end{lstlisting}

To implicira manualno slanje upita prema bazi podataka i provjeru postoji li kartica.
Kako je ovo česti problem većine aplikacija, radni okvir daje jednostavno rješenje: Tip argumenta postaviti na željeni
entitet (klasu) umjesto broja (\textit{int}, \textit{integer}).
Radni okvir pretpostavlja da se za uvjet pretrage koristi primarni ključ (\textit{ID}) jer je on jedinstven i ne može
biti \textit{null}.
Ukoliko entitet postoji, radni okvir prilaže sve informacije o tom entitetu kao argument metode i izvršava metodu.
Ukoliko entitet ne postoji, radni okvir zaustavlja izvršavanje skripte i ne poziva metodu.

Validacija podataka zahtjeva drugačije ponašanje.

\begin{lstlisting}[language=PHP]
public function rules(): array
{
    $cardUnderModification = $this->route('card');

    return [
        'name' => [
            'required',
            'string',
            'max:30',
            Rule::unique(Card::class, 'name')->ignore($cardUnderModification->id)
        ]
    ];
}
\end{lstlisting}

Kako korisnik može dati isti podatke (nije ništa promijenio) zahtjev se mora normalno izvršiti, bez ikakvih greški.
Sve što je potrebno napraviti je pri provjeri unikatnosti ignorirati karticu koja se uređuje.

Prema tablici~\ref{tab:card_functionalities} brisanje kartice je implementirano u metodi \textbf{destroy}.
Kao i kod uređivanja kartice, tip argumenta je entitet \textit{Card} kojeg radni okvir automatski ubacuje pri izvršavanju metode.
Ovisno o uspješnosti brisanja, prikladna poruka se vraća u JSON formatu.

\begin{lstlisting}[language=PHP]
public function destroy(Card $card): JsonResponse
{
    try {
        $card->delete();

        return response()->json(['data' => 'Kartica uspjesno obrisana.']);
    }
    catch (Throwable $exception) {
        return response()->json(['data' => 'Neuspjesno brisanje kartice. Pokusajte ponovo!']);
    }
}
\end{lstlisting}

\subsection{Autorizacija kartica}

Prilikom podnošenja zahtjeva sef mora poslati UID kartice koju treba autorizirati.
Ukoliko kartica postoji vraća se HTTP statusni kod 200 kao indikator uspješnog autoriziranja ili 403 kao indikator
neuspješnog autoriziranja.
U pristupni dnevnik je potrebno stvoriti novi zapis s datumom i UID-em kartice kao i (ne)uspješnost autorizacije.

Autorizacija je izložena na putanji \textbf{/api/box/authorize} kroz \textbf{POST} metodu,

\begin{lstlisting}[language=PHP]
Route::post('/box/authorize', BoxAuthorizationAction::class);
\end{lstlisting}

a implementirana je u \textbf{BoxAuthorizationAction} metodi:

\begin{lstlisting}[language=PHP]
class BoxAuthorizationAction
{
    public function __invoke(Request $request): Response
    {
        $uid = (string) $request->uid;

        if (Card::existsByUid($uid)) {
            event(new BoxAccessWasGranted($uid));
            return response()->noContent(200);
        }

        event(new BoxAccessWasProhibited($uid));
        return response()->noContent(403);
    }
}
\end{lstlisting}

Kako je autorizacijska logika zasebna klasa, koristi se posebna metoda \textit{\textunderscore\textunderscore invoke} koja omogućuje pozivanje objekta
poput običnih funkcija.
Prethodno definirana metoda \textit{existsByUid} u entitetu "Kartica" koristi se pri odlučivanju koja akcija se poduzima:
odobravanje ili odbijanje zahtjeva.
Logika zapisivanja u pristupni dnevnik nije direktno implementirana već se koristi sustav događaja (eng. \textit{event})
i osluškivača (eng. \textit{listener}).
Događaji definiraju popratne informacije dok osluškivači koriste te informacije i poduzimaju određene akcije.

\pagebreak

Dovoljna su dva događaja:
\begin{enumerate}
    \item Pristup dozvoljen
        \begin{lstlisting}[language=PHP]
class BoxAccessWasGranted
{
    private string $uid;

    public function __construct(string $uid)
    {
        $this->uid = $uid;
    }

    public function uid(): string
    {
        return $this->uid;
    }
}
        \end{lstlisting}
    \item Pristup odbijen
        \begin{lstlisting}[language=PHP]
class BoxAccessWasProhibited
{
    private string $uid;

    public function __construct(string $uid)
    {
        $this->uid = $uid;
    }

    public function uid(): string
    {
        return $this->uid;
    }
}
        \end{lstlisting}
\end{enumerate}

Oba događaja primaju \textit{UID} kao parametar i daju ga na raspolaganje osluškivačima kroz metodu \textit{uid()}.

Prije implementiranja osluškivača potrebno je stvoriti entitet "Pristupni dnevnik".
On sadrži podatke o kartici (\textit{UID}), datum i vrijeme i (ne)uspješnost pristupa.
Poput entiteta "Kartica" i ovaj entitet je vrlo jednostavan pa nema potrebe za ER dijagramom, samo migracijski kod:

\begin{lstlisting}[language=PHP]
class CreateAccessLogsTable extends Migration
{
    public function up()
    {
        Schema::create('access_logs', function (Blueprint $table) {
            $table->id();
            $table->string('uid');
            $table->timestamp('at_time');
            $table->boolean('access_granted');
        });
    }


    public function down()
    {
        Schema::dropIfExists('access_logs');
    }
}
\end{lstlisting}

Naravno, potrebna je i klasa \textbf{AccessLog} koja nasljeđuje klasu \textit{Model} da se izbjegne rad s čistim SQL-om.

\begin{lstlisting}[language=PHP]
class AccessLog extends Model
{
    protected $casts = [
        'at_time' => 'datetime',
        'access_granted' => 'boolean'
    ];
}
\end{lstlisting}

Važno je određena polja prebaciti u određene tipove podataka (atribut \textbf{\$casts}).
Kako baza podataka vraća datum i vrijeme u obliku teksta potrebno je prebaciti tip podataka u klasu koja olakšava rad s datumima:
\textit{DateTime}.
Polje \textit{at\textunderscore time} će biti prebačeno u tip \textit{DateTime}, a polje \textit{access\textunderscore granted} u tip \textit{boolean}.


Nakon definiranja modela moguće je implementirati osluškivače.

\begin{lstlisting}[language=PHP]
class LogGrantedBoxAccess
{
    public function handle(BoxAccessWasGranted $event)
    {
        $uid = $event->uid();

        $newLogEntry = new AccessLog();
        $newLogEntry->uid = $uid;
        $newLogEntry->at_time = now();
        $newLogEntry->access_granted = true;

        $newLogEntry->save();
    }
}

class LogProhibitedBoxAccess
{
    public function handle(BoxAccessWasProhibited $event)
    {
        $uid = $event->uid();

        $newLogEntry = new AccessLog();
        $newLogEntry->uid = $uid;
        $newLogEntry->at_time = now();
        $newLogEntry->access_granted = false;

        $newLogEntry->save();
    }
}
\end{lstlisting}

Oba osluškivača dijele sličnu logiku: stvore novi \textit{AccessLog} model, zapišu UID iz događaja, datum i vrijeme te
(ne)uspješnost pristupa i to pohrane u bazu podataka.
U suštini, razlika je u događaju koji osluškuju i vrijednosti polja \textit{access\textunderscore granted}.
Funkcija \textit{now()} vraća objekt tipa \textit{DateTime}, a vrijednost polja \textit{access\textunderscore granted}
je \textit{true} ili \textit{false}.
Zbog prethodno definiranog atributa \textit{\$casts} u modelu \textit{AccessLog} prilikom spremanja u bazu podataka,
radni okvir zna kako pretvoriti objekte i vrijednosti u tipove podataka koje baza podataka razumije.

\subsection{Pregled dnevnika pristupa}

Korištenjem sefa puni se pristupni dnevnik.
No, koliko god zapisa postojalo, od njih nema pretjerane koristi ako se do njih ne može doći.
Zato valja izložiti te podatke grafičkom sučelju.

Potrebno je vratiti sve zapise dnevnika pristupa u JSON formatu sortirane silazno po datumu stvaranja (prvo najnoviji zapisi).
Zapisi su dostupni na putanji \textbf{/api/access-log} kroz \textbf{GET} metodu,

\begin{lstlisting}[language=PHP]
Route::get('/access-log', FetchAccessLogAction::class);
\end{lstlisting}

a implementirana je u \textbf{FetchAccessLogAction} metodi:

\begin{lstlisting}[language=PHP]
class FetchAccessLogAction
{
    public function __invoke(Request $request): JsonResponse
    {
        $accessLogEntries = AccessLog::query()->orderByDesc('at_time')->get();

        return response()->json(['data' => $accessLogEntries]);
    }
}
\end{lstlisting}

Važno je naglasiti da se zapisi ne povezuju s entitetom "Kartica" već je to delegirano na grafičko sučelje.
Kako korisnik može izmjenjivati i brisati kartice, nakon bilo koje akcije bilo bi potrebno osvježiti podatke iz dnevnika
pristupa što generira nepotreban mrežni promet.
S obzirom da je logika povezivanja ovih entiteta vrlo jednostavna (oba entiteta moraju sadržavati isti \textit{UID})
sigurno je implementirati funkcionalnost u grafičkom sučelju.

\subsection{Autorizacija korisnika}

Sadržaj dnevnika pristupa i upravljanje karticama nije javno dostupno.
Korisnik je dužan autorizirati se pomoću svojih pristupnih podataka (korisničko ime i lozinka) prije poduzimanja bilo
kojih drugih radnji.

Autorizacija korisnika je dostupna na putanji \textbf{/api/login} kroz \textbf{POST} metodu.

\begin{lstlisting}[language=PHP]
Route::post('login', LoginAction::class);
\end{lstlisting}

Prije implementiranja autorizacije potrebno je pohraniti korisničke pristupne podatke.
Za to će poslužiti jednostavan entitet "Korisnik" koji sadrži podatke o korisničkom imenu i lozinki.
Poput prethodnih entiteta i ovaj entitet je vrlo jednostavan pa nema potrebe za ER dijagramom, samo migracijski kod:

\begin{lstlisting}[language=PHP]
class CreateUsersTable extends Migration
{
    public function up()
    {
        Schema::create('users', function (Blueprint $table) {
            $table->id();
            $table->string('username')->unique();
            $table->string('password');
            $table->timestamps();
        });
    }

    public function down()
    {
        Schema::dropIfExists('users');
    }
}
\end{lstlisting}

Potreban je i model \textbf{User}, no ovaj put se ne nasljeđuje klasa \textit{Model} već klasa \textit{Authenticatable}.
Ta klasa sadrži određene metode potrebne za autoriziranje korisnika pa radni okvir pomoću nje zna kako autorizirati korisnika
i stvoriti sesiju.

\begin{lstlisting}[language=PHP]
class User extends Authenticatable
{
}
\end{lstlisting}

Radni okvir sadrži korisnu metodu \textbf{Auth::attempt} koja prima jedan parametar tipa polje.
U polje se postave vrijednosti korisničkog imena i lozinke koje je poslao korisnik i ako su podaci točni, korisnik je autoriziran.
Ovisno o ishodu metode vraća se prikladna poruka u JSON formatu.

\begin{lstlisting}[language=PHP]
class LoginAction extends Controller
{
    public function __invoke(Request $request): JsonResponse
    {
        $username = $request->username;
        $password = $request->password;

        if (Auth::attempt(['username' => $username, 'password' => $password]) === false) {
            return response()->json(['message' => 'Korisnicko ime i/ili lozinka su ne ispravni.'], 401);
        }

        return response()->json(['message' => 'Uspjesna prijava']);
    }
}
\end{lstlisting}
